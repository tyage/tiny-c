\documentclass[a4paper,12pt]{jarticle}
\usepackage{listings}
\usepackage{url}
\title{計算機科学実験及演習3 中間レポート1}
\author{1029-24-9540 山崎啓太郎}
\begin{document}
\lstset{numbers=left,basicstyle=\small}
\maketitle

\section{課題3}
\subsection{回答}
課題3の仕様を持つTiny Cパーサを作成しました。

\subsection{プログラムを置いたディレクトリパス}
\verb|$HOME/dev/tiny-c/task3/haskell/| にHaskellで作成した場合のソースコードを置いてあります。

以下の説明はHaskellで作成したソースコードの説明になります。

\verb|$HOME/dev/tiny-c/task3/c/| にC言語で作成した場合のソースコードを置いてあります。

(\verb|$HOME = /export/home/012/a0121573|)

\subsection{各プログラムの設計方針}
\begin{description}
  \item[Main.hs] ファイルを読み込んでパーサにかけます
  \item[Parser.hs] 課題3の仕様に合わせたTiny Cパーサになります
  \item[AST.hs] Parser.hsで使用する型に必要となる抽象構文木になります
\end{description}

\subsection{各部の説明}
\begin{description}
  \item[Main.hs] パーサを呼び出します
  \item[Parser.hs] 課題3の仕様に合わせたTiny Cパーサになります
  \item[AST.hs] 抽象構文木になります
  \item[test.c] パーサのテスト用Tiny Cのコードです
\end{description}

\subsection{感想}
課題3の仕様を一つずつ確認してパーサを構築していく手順に時間がかかりました。

\section{課題4}
\subsection{回答}
課題4の実行例と似た形式で出力するパーサを作成しました。

構文的に誤りのない場合は型を利用して構文木が出力され、構文的に誤りのある場合は例えば以下の様に出力されます。

以下の様なソースコードに対し

\begin{verbatim}
int a b;
\end{verbatim}

以下の様な出力がされます。

\begin{verbatim}
"TinyC" (line 1, column 7):
unexpected "b"
expecting "("
\end{verbatim}

\subsection{プログラムを置いたディレクトリパス}
\verb|$HOME/dev/tiny-c/task3/haskell/| にソースコードを置いてあります。

\verb|$HOME/dev/tiny-c/task4/| にテストコードを置いてあります。

(\verb|$HOME = /export/home/012/a0121573|)

\subsection{各プログラムの設計方針}
課題3と同様

\subsection{各部の説明}
以下が課題3と異なります。

\begin{description}
  \item[test.tc] 実行例に使われていたTiny Cプログラムのファイル
\end{description}

\subsection{感想}
打ち間違い(declaratorやdeclaration)などでパーサが正常に機能しないことがいくつかあり、修正に時間がかかった。

\end{document}
